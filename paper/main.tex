% arXiv-ready minimal article
\documentclass[11pt]{article}
\usepackage[a4paper,margin=1in]{geometry}
\usepackage{amsmath,amssymb}
\usepackage{graphicx}
\usepackage{hyperref}
\usepackage[numbers,sort&compress]{natbib}
\usepackage{authblk}

% Macros
\newcommand{\ulwem}{ULW-EM}
\newcommand{\fdb}{FDB}

\title{Ultra Long Wavelength Electromagnetism as Apparent Gravity: A Multi-Scale Model and Galaxy Rotation Curve Tests}
\author[1]{First Last}
\affil[1]{Affiliation, City, Country}
\date{\\Draft: \today}

\begin{document}
\maketitle

\begin{abstract}
We propose that ultra long wavelength electromagnetism (\ulwem) gives rise to an apparent gravitational effect (\fdb) that explains galaxy-scale discrepancies without dark matter. We formulate several parametric and multi-scale models and evaluate them against SPARC/THINGS rotation curves and MaNGA-derived mass distributions.
\end{abstract}

\section{Introduction}
Observed galaxy rotation curves deviate from the expectations of general relativity (GR) with baryonic matter alone. We explore whether an ultra long wavelength electromagnetic field (\ulwem) can induce a bias that manifests as apparent gravity, which we call Future Decoherence Bias (\fdb), reproducing these discrepancies without invoking particle dark matter.

\section{Theory}
\label{sec:theory}

We recast ULW--EM as a \emph{future\,fixation bias}: conditioning dynamics on
the occurrence of a future ``record'' event $E$ leads to apparent attraction
without mechanical work. Mathematically this is a Doob $h$\,transform with an
\emph{information potential}
\begin{equation}
  U_{\text{info}}(\vec x,t) = \kappa\,\ln h(\vec x,t),\qquad
  h(\vec x,t)=\mathbb P(E\mid \vec x_t = \vec x),
\end{equation}
such that along typical paths
\begin{equation}
  \frac{d}{dt}\bigl[K(\vec x,\dot{\vec x})+U_{\text{info}}(\vec x,t)\bigr]=0,
\end{equation}
and the apparent force law remains
\begin{equation}
  m\,\ddot{\vec x} = -\nabla U_{\text{info}} = -\kappa\,\nabla\ln h.
\end{equation}

For a baseline Markov process with generator $\mathcal L$, the conditioned
dynamics has generator $\mathcal L^{(E)} f = h^{-1}\,\mathcal L(h f)$. In
diffusion limits this yields an effective drift $b_{\rm eff}=b+2D\,\nabla\ln h$.

\subsection{Weak\,field normalization and lensing}
\label{subsec:normalization}

Choosing $\kappa = m c^2$ and $h = \exp(\Phi_{\rm eff}/c^2)$ gives
$U_{\text{info}} = m\,\Phi_{\rm eff}$ and $\ddot{\vec x} = -\nabla\Phi_{\rm eff}$,
so inertial mass cancels as required by the weak equivalence principle. For
light with momentum $p$, set $U^{(\gamma)}_{\text{info}}=(p/c)\,\Phi_{\rm eff}$.
The weak\,field deflection is then
\begin{equation}
  \hat\alpha = 2 \int \nabla_\perp\!\left(\frac{\Phi_{\rm eff}}{c^2}\right) ds,
\end{equation}
which reproduces the point\,mass result $\hat\alpha=4GM/(b c^2)$.

\subsection{Geometric bias: a spherical\,shell example}
\label{subsec:shell}

In the far field, conditional detectability scales as $h\propto 1/d^2$, so
$\nabla\ln h = -2\nabla\ln d$ points toward the observer. For a thin shell of
radius $r$ seen from distance $R>r$, the posterior mean along the line of sight
conditioned on a detection is
\begin{equation}
  \langle z\rangle_{\rm click}
  = r\,\frac{\tfrac{R^2+r^2}{2Rr}\ln\!\frac{R+r}{R-r}-1}
                {\ln\!\frac{R+r}{R-r}}
  = \frac{2}{3}\,\frac{r^2}{R} + \mathcal O\!\left(\frac{r^4}{R^3}\right)\!>\!0.
\end{equation}

\subsection{Model freedom in $\Phi_{\rm eff}$}
\label{subsec:phi-eff}

We treat $\Phi_{\rm eff}$ as a design layer constrained by observations:
minimal $\Phi_{\rm eff}$ is the Newtonian potential from baryons; extensions
use linear kernels $\mu(k)$ (e.g. Yukawa sums) or non\,linear
$\nu(a_N/a_0)$ forms (MOND\,like) to match the RAR/BTFR and lensing while
respecting solar\,system bounds.

\subsection{Microscopic origin and sum rule for $\kappa$ and $C$}
\label{subsec:kappaC}

At the microscopic level we parameterize ULW emission and recording by
per\,mass emission rate $\phi_{\rm emit}(\omega)$ [s$^{-1}$Hz$^{-1}$kg$^{-1}$]
and per\,mass effective ``recording'' cross section $\sigma_M(\omega)$ [m$^2$kg$^{-1}$].
In the isotropic far\,field the hazard kernel induced by a single recordable
quantum decays as $K(R)=C/R^2$ with
\begin{equation}
  \boxed{\ C\;=\;\frac{1}{4\pi}\int_0^\infty\!\phi_{\rm emit}(\omega)\,\sigma_M(\omega)\,d\omega\ }\,.
\end{equation}
The information\,energy conversion scale is fixed by Lorentz invariance,
\begin{equation}
  \boxed{\ \kappa = m c^2\,,\qquad \kappa^{(\gamma)} = p c\ }\,.
\end{equation}
Combining these with the Newtonian limit yields the necessary and sufficient
condition
\begin{equation}
  \boxed{\ \frac{\kappa}{m}\,\frac{C}{c} = G\ }\,\quad\Rightarrow\quad
  \kappa = m c^2\ \Longrightarrow\ \boxed{\ C = \frac{G}{c}\ }\,.
\end{equation}
We refer to this as a ``sum\,rule'' over ULW emission and recording: the
frequency integral of $\phi_{\rm emit}\,\sigma_M$ per mass equals $4\pi G/c$.
Weak equivalence (composition independence per mass), reciprocity
(emission/recording symmetry) and energy accounting are simultaneously
respected; observational constraints then enter through the shapes of
$\phi_{\rm emit}(\omega)$, $\sigma_M(\omega)$, while the overall scale is
fixed by $C=G/c$.

\section{Models}
FDB stands for Future Decoherence Bias.

\paragraph{FDB1} Single parameter $\theta$:
\begin{equation}
 g = g_{\rm GR} (1+\theta^2).
\end{equation}
\paragraph{FDB2} Two parameters $(a,b)$:
\begin{equation}
 g = g_{\rm GR} + a\, \tanh\!\left(\frac{r}{b}\right).
\end{equation}
\paragraph{FDB3} Three parameters $(a,b,c)$:
\begin{equation}
 g = g_{\rm GR} + a\, \bigl(1-e^{-r/b}\bigr)^{c}.
\end{equation}
\paragraph{FDBL (3D multi-scale)} Octree over the 3D mass distribution with each depth mapping to a wavelength band and weight $w(\lambda)$.

\section{Data}
We use MaNGA (SDSS DR17/18) LOGCUBE and MAPS products to estimate 3D mass distributions, the SPARC catalog for rotation curves, and THINGS HI maps for high-resolution checks. Directory layout and acquisition scripts are in `data/` and `scripts/`.

\section{Methods}
\subsection{ULW--EM solver (minimal)}
We use a phenomenological builder for the effective potential $\Phi_{\rm eff}$
consistent with the information\,potential formulation (\S\ref{sec:theory}).
Concretely, we solve a Yukawa\,screened Poisson equation
$(\nabla^2-\lambda^{-2})\,\phi=\beta\,j_{\rm EM}$ on a 2D grid by FFT, set
$\Phi_{\rm eff}=\eta\,\phi$, and compute the apparent acceleration
$\vec g_{\rm app}=-\nabla\Phi_{\rm eff}$. Circular averages yield
$g_R(R)$ and $v_c(R)=\sqrt{R\,g_R}$. A reference implementation is provided in
\texttt{src/models/fdbl.py}. A dependency\,free demo (synthetic exponential
disk to SVG) is available as \texttt{scripts/demo\_fdbl.py}.

Inputs for $j_{\rm EM}$ will be estimated from WISE (W1/W2) and GALEX
(FUV/NUV) in subsequent stages; here we use an exponential disk trial to
illustrate the response and produce a figure for qualitative assessment.
We integrate MaNGA LOGCUBE and MAPS to estimate baryonic mass distributions, derive rotation curves, and normalize observed velocities ($V_{\rm obs}$). Parameter inference uses joint fits across multiple galaxies with coverage targets and regularization informed by RJ spectral ratios.

\section{Results}
On a SPARC sample, galaxy-by-galaxy fits show substantial
improvements moving from GR(noDM) to FDB3, and a joint fit with
shared $(a,b,c)$ also reduces the overall $\chi^2_{\nu}$.

\begin{figure}[t]
  \centering
  \includegraphics[width=0.95\linewidth]{figures/
  sota_improvement_hist.png}
  \caption{Distribution of improvement factors
  ($\chi^2_\nu$ of GR divided by FDB3) across SPARC galaxies.}
  \label{fig:hist}
\end{figure}

\begin{figure}[t]
  \centering
  \includegraphics[width=0.95\linewidth]{figures/
  sota_redchi2_scatter.png}
  \caption{Per-galaxy $\chi^2_\nu$ for GR(noDM) versus FDB3.
  Points below the diagonal indicate FDB3 improvements.}
  \label{fig:scatter}
\end{figure}

\begin{figure}[t]
  \centering
  \includegraphics[width=0.95\linewidth]{figures/
  sota_vr_panel.png}
  \caption{Representative rotation curves and best-fit models
  (GR vs. FDB3).}
  \label{fig:vrpanel}
\end{figure}

\begin{figure}[t]
  \centering
  \includegraphics[width=0.95\linewidth]{figures/
  compare_fit_DDO154_shared.png}
  \caption{DDO154 with shared $(a,b,c)$ from the global fit and
  galaxy-specific mass-to-light parameters.}
  \label{fig:ddo154shared}
\end{figure}

\section{Discussion}
We discuss physical plausibility, the role of emission temperature via RJ weighting, limitations including potential unit mismatches in GR baselines, and implications for large-scale phenomena if \ulwem\ induces time-lag effects $\Delta t = \lambda/c$.

\section{Conclusion}
\ulwem-based apparent gravity offers a unified, multi-scale avenue to explain galaxy rotation curves without particle dark matter. Next steps include implementing and validating the 3D octree model (FDBL), enforcing RJ-based regularization, and expanding to higher-resolution galaxies (e.g., NGC 3198, NGC 2403).


\section*{Acknowledgments}
We thank collaborators and data providers (SDSS MaNGA, SPARC, THINGS).

\bibliographystyle{unsrtnat}
\bibliography{references}

\end{document}
