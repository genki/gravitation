\section{Methods}
\subsection{ULW--EM solver (minimal)}
We use a phenomenological builder for the effective potential $\Phi_{\rm eff}$
consistent with the information\,potential formulation (\S\ref{sec:theory}).
Concretely, we solve a Yukawa\,screened Poisson equation
$(\nabla^2-\lambda^{-2})\,\phi=\beta\,j_{\rm EM}$ on a 2D grid by FFT, set
$\Phi_{\rm eff}=\eta\,\phi$, and compute the apparent acceleration
$\vec g_{\rm app}=-\nabla\Phi_{\rm eff}$. Circular averages yield
$g_R(R)$ and $v_c(R)=\sqrt{R\,g_R}$. A reference implementation is provided in
\texttt{src/models/fdbl.py}. A dependency\,free demo (synthetic exponential
disk to SVG) is available as \texttt{scripts/demo\_fdbl.py}.

Inputs for $j_{\rm EM}$ will be estimated from WISE (W1/W2) and GALEX
(FUV/NUV) in subsequent stages; here we use an exponential disk trial to
illustrate the response and produce a figure for qualitative assessment.
We integrate MaNGA LOGCUBE and MAPS to estimate baryonic mass distributions, derive rotation curves, and normalize observed velocities ($V_{\rm obs}$). Parameter inference uses joint fits across multiple galaxies with coverage targets and regularization informed by RJ spectral ratios.
