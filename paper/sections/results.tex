\section{Results}
On a SPARC sample, galaxy-by-galaxy fits show substantial
improvements moving from GR(noDM) to FDB3, and a joint fit with
shared $(a,b,c)$ also reduces the overall $\chi^2_{\nu}$.

\begin{figure}[t]
  \centering
  \includegraphics[width=0.95\linewidth]{figures/
  sota_improvement_hist.png}
  \caption{Distribution of improvement factors
  ($\chi^2_\nu$ of GR divided by FDB3) across SPARC galaxies.}
  \label{fig:hist}
\end{figure}

\begin{figure}[t]
  \centering
  \includegraphics[width=0.95\linewidth]{figures/
  sota_redchi2_scatter.png}
  \caption{Per-galaxy $\chi^2_\nu$ for GR(noDM) versus FDB3.
  Points below the diagonal indicate FDB3 improvements.}
  \label{fig:scatter}
\end{figure}

\begin{figure}[t]
  \centering
  \includegraphics[width=0.95\linewidth]{figures/
  sota_vr_panel.png}
  \caption{Representative rotation curves and best-fit models
  (GR vs. FDB3).}
  \label{fig:vrpanel}
\end{figure}

\begin{figure}[t]
  \centering
  \includegraphics[width=0.95\linewidth]{figures/
  compare_fit_DDO154_shared.png}
  \caption{DDO154 with shared $(a,b,c)$ from the global fit and
  galaxy-specific mass-to-light parameters.}
  \label{fig:ddo154shared}
\end{figure}
