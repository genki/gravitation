\section{Theory}
\label{sec:theory}

We recast ULW--EM as a \emph{future\,fixation bias}: conditioning dynamics on
the occurrence of a future ``record'' event $E$ leads to apparent attraction
without mechanical work. Mathematically this is a Doob $h$\,transform with an
\emph{information potential}
\begin{equation}
  U_{\text{info}}(\vec x,t) = \kappa\,\ln h(\vec x,t),\qquad
  h(\vec x,t)=\mathbb P(E\mid \vec x_t = \vec x),
\end{equation}
such that along typical paths
\begin{equation}
  \frac{d}{dt}\bigl[K(\vec x,\dot{\vec x})+U_{\text{info}}(\vec x,t)\bigr]=0,
\end{equation}
and the apparent force law remains
\begin{equation}
  m\,\ddot{\vec x} = -\nabla U_{\text{info}} = -\kappa\,\nabla\ln h.
\end{equation}

For a baseline Markov process with generator $\mathcal L$, the conditioned
dynamics has generator $\mathcal L^{(E)} f = h^{-1}\,\mathcal L(h f)$. In
diffusion limits this yields an effective drift $b_{\rm eff}=b+2D\,\nabla\ln h$.

\subsection{Weak\,field normalization and lensing}
\label{subsec:normalization}

Choosing $\kappa = m c^2$ and $h = \exp(\Phi_{\rm eff}/c^2)$ gives
$U_{\text{info}} = m\,\Phi_{\rm eff}$ and $\ddot{\vec x} = -\nabla\Phi_{\rm eff}$,
so inertial mass cancels as required by the weak equivalence principle. For
light with momentum $p$, set $U^{(\gamma)}_{\text{info}}=(p/c)\,\Phi_{\rm eff}$.
The weak\,field deflection is then
\begin{equation}
  \hat\alpha = 2 \int \nabla_\perp\!\left(\frac{\Phi_{\rm eff}}{c^2}\right) ds,
\end{equation}
which reproduces the point\,mass result $\hat\alpha=4GM/(b c^2)$.

\subsection{Geometric bias: a spherical\,shell example}
\label{subsec:shell}

In the far field, conditional detectability scales as $h\propto 1/d^2$, so
$\nabla\ln h = -2\nabla\ln d$ points toward the observer. For a thin shell of
radius $r$ seen from distance $R>r$, the posterior mean along the line of sight
conditioned on a detection is
\begin{equation}
  \langle z\rangle_{\rm click}
  = r\,\frac{\tfrac{R^2+r^2}{2Rr}\ln\!\frac{R+r}{R-r}-1}
                {\ln\!\frac{R+r}{R-r}}
  = \frac{2}{3}\,\frac{r^2}{R} + \mathcal O\!\left(\frac{r^4}{R^3}\right)\!>\!0.
\end{equation}

\subsection{Model freedom in $\Phi_{\rm eff}$}
\label{subsec:phi-eff}

We treat $\Phi_{\rm eff}$ as a design layer constrained by observations:
minimal $\Phi_{\rm eff}$ is the Newtonian potential from baryons; extensions
use linear kernels $\mu(k)$ (e.g. Yukawa sums) or non\,linear
$\nu(a_N/a_0)$ forms (MOND\,like) to match the RAR/BTFR and lensing while
respecting solar\,system bounds.

\subsection{Microscopic origin and sum rule for $\kappa$ and $C$}
\label{subsec:kappaC}

At the microscopic level we parameterize ULW emission and recording by
per\,mass emission rate $\phi_{\rm emit}(\omega)$ [s$^{-1}$Hz$^{-1}$kg$^{-1}$]
and per\,mass effective ``recording'' cross section $\sigma_M(\omega)$ [m$^2$kg$^{-1}$].
In the isotropic far\,field the hazard kernel induced by a single recordable
quantum decays as $K(R)=C/R^2$ with
\begin{equation}
  \boxed{\ C\;=\;\frac{1}{4\pi}\int_0^\infty\!\phi_{\rm emit}(\omega)\,\sigma_M(\omega)\,d\omega\ }\,.
\end{equation}
The information\,energy conversion scale is fixed by Lorentz invariance,
\begin{equation}
  \boxed{\ \kappa = m c^2\,,\qquad \kappa^{(\gamma)} = p c\ }\,.
\end{equation}
Combining these with the Newtonian limit yields the necessary and sufficient
condition
\begin{equation}
  \boxed{\ \frac{\kappa}{m}\,\frac{C}{c} = G\ }\,\quad\Rightarrow\quad
  \kappa = m c^2\ \Longrightarrow\ \boxed{\ C = \frac{G}{c}\ }\,.
\end{equation}
We refer to this as a ``sum\,rule'' over ULW emission and recording: the
frequency integral of $\phi_{\rm emit}\,\sigma_M$ per mass equals $4\pi G/c$.
Weak equivalence (composition independence per mass), reciprocity
(emission/recording symmetry) and energy accounting are simultaneously
respected; observational constraints then enter through the shapes of
$\phi_{\rm emit}(\omega)$, $\sigma_M(\omega)$, while the overall scale is
fixed by $C=G/c$.
